Qui di seguito è riportato il \verb|dump| in SQL per la creazione delle tabelle all'interno di PostgreSQL
\begin{lstlisting}[language=SQL]
CREATE TABLE "repository" (
    id uuid PRIMARY KEY NOT NULL,
    url varchar(255) UNIQUE NOT NULL,
    created_at timestamp NOT NULL DEFAULT NOW(),
    updated_at timestamp NOT NULL DEFAULT NOW(),
    uploader_ip varchar(21) NOT NULL
);

CREATE TABLE "email"(
    email varchar(120) PRIMARY KEY NOT NULL,
    hash_md5 varchar(32) UNIQUE NOT NULL
);

CREATE TABLE "commit" (
    hash varchar(40) PRIMARY KEY NOT NULL,
    tree varchar(40) REFERENCES commit(hash) NULL,
    text text NOT NULL,
    date timestamptz NOT NULL,
    author_email varchar(120) REFERENCES email(email) NOT NULL,
    author_name varchar(120) NOT NULL,
    committer_email varchar(120) REFERENCES email(email) NOT NULL,
    committer_name varchar(120) NOT NULL,
    repository_url varchar(256) REFERENCES repository(url) NOT NULL 
);

CREATE TABLE "branch" (
    id uuid PRIMARY KEY NOT NULL,
    name varchar(120) NOT NULL,
    repository_id uuid REFERENCES repository(id) NOT NULL,
    head varchar(40) REFERENCES commit(hash) NULL
);
\end{lstlisting}

\section{Trigger}
Qui di seguito si viene implementato un trigger che aggiorna automaticamente la data di \verb|updated_at| ogni qual volta si viene modificata una \verb|Repository|.
\begin{lstlisting}[language=SQL]
CREATE OR REPLACE FUNCTION update_repository_last_updates() RETURNS TRIGGER	AS $$
BEGIN
	NEW."updated_at" = NOW();
	RETURN NEW;
END;
$$ LANGUAGE PLPGSQL;

CREATE TRIGGER update_repository_t BEFORE UPDATE ON repository
FOR EACH ROW
EXECUTE PROCEDURE update_repository_last_updates();
\end{lstlisting}

\section{Top authors}
Lato client si sarebbero potuti esaminare i dati presi dai commit ma, per velocizzare il tutto lato client, ho preferito inserire una query lato server con un'endpoint per trovare la TOP 7 degli sviluppatori con più commit di cui sono autori. \textit{Ho preferito non considerare i committers perché, se presi da GitHub, automaticamente il committer sarà sempre noreply@github.com}.

\begin{lstlisting}[language=SQL]
SELECT COUNT(hash) as num, author_email, author_name
FROM commit
GROUP BY author_email, author_name
ORDER BY COUNT(hash) DESC
\end{lstlisting}
