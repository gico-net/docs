In conclusione posso dire che la webapp non è completa affatto. Come lei, qualsiasi altro software FOSS: esso può migliorare e mutare. Non posso dire con certezza che sarà mantenuto con costanza, ma ci proverò.


Una cosa che vorrei introdurre sono i \textit{commenti}: essi potranno riferirsi alla repository o al singolo commit.

Ci sono svariate altre modifiche per l'app backend e frontend che si potrebbero e dovrebbero fare, ma preferisco non includerle in questo documento. Ho citato prima i commenti perché sono una funzionalità lato backend che si interlacciano al database.\\\\
Il trigger mostrato nella \verb|Sezione 5.1| si riferisce al linguaggio SQL usato da PostgreSQL, lo si può reinterpretare senza la divisione con la procedura per dar luogo a un SQL più simile a quello visto durante il corso di Basi di Dati.\\\\
Le immagini sono state realizzate con un programma di terze parti\footnote{https://app.diagrams.net/}. Questo documento è stato realizzato in \LaTeX\ mediante Gummi\footnote{https://github.com/alexandervdm/gummi}.
