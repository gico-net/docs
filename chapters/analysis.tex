Dalla creazione di Git sono stati creati svariati client web, tra cui il più famoso GitHub. Qualunque sviluppatore in erba - o una persona che si definisce tale - ha un account GitHub o passa il tempo a smanettare la sezione \textit{Explore} per scoprire nuovi progetti. In questo caso io consiglio di guardare le repository \textit{awesome-\textbf{stack}} per scoprire progetti, più o meno famosi, che usano quel determinato stack. Gli sviluppatori più smanettoni hanno un'istanza di un client opensource instanziata sulla propria macchina (che sia GitLab, Gitea, CGit o qualsiasi altro).\\
Ma rimaniamo in tema GitHub, dato che, anche se è proprietario, e questo turba alcune persone, resta il miglior client web Git per pubblicizzare il proprio progetto. In continua evoluzione, da qualche mese oramai, la lista dei \textit{contributors} si trova in bella vista nella sidebar laterale. È una buona prassi, da parte di tutti, "stalkerare" i contributors per vedere se hanno altri progetti interessanti a cui hanno contribuito. \\\\
Qui nasce proprio Gico, un software dove poter caricare le repositories delle altre persone e salvare tutti i commit in un database centralizzato. In questo modo si possono filtrare i commit per autore e vedere a quale repositories ne fanno parte. Si "stalkera" quindi nel medesimo modo in cui si fa nei social network, ma in modo non ossessivo e per una buona causa: la curiosità di trovare altri progetti interessanti da contribuire/usare.\\
Per come è progettato, Git salva, all'interno dei commit, l'email e il nome dell'autore e del committer. Gico si occuperà di esaminare il log della repository salvando il commit e alcuni dei suoi dati utili a noi.
\section{Glossario}
Qui di seguito una lista delle parole chiavi impiegate all'interno del database.\\
Come si può notare non c'è il termine "Utente" proprio perché non esiste, all'interno di Git, una concezione tale.
\begin{center}
\begin{tabular}{ |c|c|c|c|c| } 
\hline
Termine & Descrizione & Sinonimi & Termini collegati \\
\hline
Repository & Codice sorgente & Sourcecode, Progetto & \\
Branch & Flusso di modifiche & Diramazione & Repository \\
Commit & Modifica alla sourcecode & Modifica & Branch, Repository \\
Autore & L'autore delle modifiche & Programmatore & Commit \\
Committer & L'autore del commit & Code reviewer & Commit \\
\hline
\end{tabular}
\end{center}